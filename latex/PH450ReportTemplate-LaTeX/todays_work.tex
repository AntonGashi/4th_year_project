   Where $\sigma_{PSF}$ is the standard deviation of the point spread function (PSF)
   for the microscope, and $N$ is the number of detected photons per florescent event.
   Since the relationship of equation \ref{PSFequation} is approximately the inverse square root, 
   the more detected photons the lower the uncertainty becomes.\cite{DEMPSEY2013561}
   One of the ways to do this is by using the Nyquist-Shannon sampling theorem,
   this states that any detail in a measurement that is smaller than twice the size of the 
   average label to label distance can reliably be resolved.\cite{tinnefeld2015far}
   The Nyquist resolution limit can be written formally as; 
   \begin{equation}\label{resolution_equation}
     \textrm{Nyquist resolution limit} \/\ =\frac{2}{N^{\frac{1}{D}}}
   \end{equation}

   Where $N$ is the density of labels, and $D$ is the dimensionality. 
   Labels in this specific case refers to each individual fluorescent event that is 
   detected by the camera, the density of labels is sometimes constrained by the 
   molecules being detected as if the structure that's trying to be detected is too
   sparse the uncertainty remains relatively high. 
   For example if the resolution was required to be 20nm in 
   one dimension then fluorophores have to be separated at least 10nm apart at a density of 
   $~10^4\mu m^{-2}$ at a minimum to achieve it.
   The final resulting uncertainty of the image is therefor the maximum of either equation (\ref{PSFequation}) or (\ref{resolution_equation}). 


   \subsubsection{Fluorophores} % (fold)
   \label{ssub:Fluorophores}
   
   Single molecule super-resolution demands the switching of fluorophores
   stochastically, this can be done reversibly or irreversibly. Reversible switching fluorophores
   are photo-activated and emit light until it becomes non-fluorescent again unless reactivated.  
   Irreversible switching fluorophores either start in the off state or the on state, if it 
   starts in an off state it can be photo-activated and turned on then after a period of 
   time it becomes bleached. If it starts switched on then it can be further excited 
   and transition to a red-shifted state.\cite{van2011single}

   According to equation \ref{resolution_equation} the resolution is increased 
   if the number density $N$ is increased, although in order to locate or find a spot 
   in a diffraction limited image the spots need to be spaced apart. 
   Therefor the rate at which populations are on or off can be denoted by $k_{on}$ and 
   $k_{off}$, also the rate at which the fluorophores switch to the on and off states matter 
   and are denoted as $\tau_{on}$ and $\tau_{off}$. The switch rates are directly linked with 
   how many excitations can be detected at once and thus are linked with the ability to 
   accurately find the location of each spot. Ratios between on and off states are denoted as 
   $r=\frac{k_{off}}{k_{on}}=\frac{\tau_{off}}{\tau_{on}}$, the larger the ratio, r, the more 
   accurate results can be up until a point.
