  A reason for the centroiding method and triangle method used in this paper being slower that others may 
  be due to the fact that the code for this paper was written in python, where as the centroiding method used in 
  delabie et al. was written in C$++$. Elapsed time test results taken from a website that poses the question 
  Which programming language is the fastest?\cite{bagley}, range from a $\approx 2.5\rightarrow 95$ times 
  less time taken for C$++$ over python. This is however dependant on what is being ran, for example the largest 
  gap in elapsed time was from a symplectic-integrator that modeled planets and the smallest gap was from a program 
  that takes in strings of amino acid protein sequences and formats them.   Another reason for this papers shortfall in computational time taken may be due to the use of python 
  packages used, this includes pandas, matplotlib, scipy, tifffile and others need to be called from 
  a different file. An improvement could be defining the functions that are used in the same file they are 
  being used in. This means that what specific operations 
  are carried out on either language play a role in how fast the program runs, in part due to the fact that 
  python is an interpreted language whilst C$++$ is a compiled language. Centroiding is also 
  considered to be the fastest spot finding method.\cite{delabie2014accurate}  To conclude a spot finding method was created and others were replicated in order to compare 
  accuracy and time taken.   To conclude a spot finding method was created and others were replicated in order to compare 
  accuracy and time taken. In section \ref{sec:intro} important concepts about sub-pixel localisation 
  were discussed to give an overarching view what it is and why this paper was trying to improve it 
  in some manner. In section \ref{sec:Methods} the procedures that were carried out like centroiding 
  and the Gaussian fitting method were explained step by step to understand the apparent costs and benefits 
  of well understood and common place algorithms. Furthermore a new method was created and discussed in order 
  to align with the motivation behind the study of trying to produce a method that is still fast in terms 
  of time taken to give an answer but also more accurate than an algorithm like centroiding. In section \ref{sec:Results}
  the centroiding and triangle algorithms were tested with spot data provided, for the simulated perfect data this resulted 
  in the centroiding algorithm predictably being faster (160 times) and being more accurate ($\approx 19\rightarrow 1.25$ times) 
  except from the case of the 3x3 box size where the triangle method has better accuracy ($\approx 1.33\rightarrow 2.6$ times).
  In the case of the simulated noisy data, again, the centroiding method takes the advantage with results more 
  accurate ($\approx 1.5\rightarrow 2$ times).

