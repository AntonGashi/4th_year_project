       Once the triangle algorithm was implemented it was tested on TIFF stacks of 100 images 
    of perfect spot data. The absolute error (Euclidean distance) was calculated from each 
    axis and plotted as seen in figure \ref{fig:single_histo} A. In image B a histogram 
    was produced so that the spread was more visible.

   For the centroiding method box sizes of 3, 5, 7 and 11 were used in the production 
  of the histograms pertaining to the absolute error, this was done for the noisy data as 
  well. The rest of the data not presented in the main body of text is available in the appendix 
  section (\ref{sec:appendix}).

 The results of the centroiding algorithm in figure \ref{fig:box_3} show a fairly poor 
  level of accuracy, ranging from a $0\rightarrow 1.5$ pixel error approximately. This lack of accuracy 
  is especially pronounced the larger the radii of spot, the reason for this increasing 
  error is due to the spots being larger in the same sized 50x50 images. The radii of the 
  airy disk that was used to generate the spots is 'physically' larger than the box size 
  that the centroiding is using. This holds true for the radii that are larger than R1.5 
  as the box size's diameter, as well as the diameter of the airy disk, is 3 pixels. The further 
  increase of inaccuracy seen in figure \ref{fig:box_3} R4.00, R5.66 and R8.00 is due to the 
  3x3 box being placed roughly where the centre of the spot already is, then the algorithm is 
  ran which in essence takes the average of the intensities. This 3x3 box is not given a reasonable 
  range of intensities to average over thus the method will be less likely to give the correct answer,
  this instead gives a range of answers from $0\rightarrow 1.5$ pixels approximately as if the 
  local maximum finder roughly gives an answer the centroiding algorithm will only give and answer 
  that is 1.5 pixels off of that guess. Since in this perfect data that only has one spot per image, the 
  local maximum finder will predict where the spot is to a single pixel of accuracy thus the maximum 
  error seen should indeed be fractionally larger than 1.5 pixels as observed.

  In figure \ref{fig:box_3_noise} noisy spot data was provided to test out the algorithms on, 
  these spot were the same as before but with read noise at 1 electron, average photon number 
  per spot at 2000 and dark current at 0.1 electron per pixel and cg? Also the number of radii 
  were reduced, R1.00 and R1.41 were removed. These noise values were chosen 
  to replicate real world noise levels in order to test all algorithms in isolation from any other 
  variables so that the spot finding methods could be objectively observed. 

  The noisy data for the 3x3 centroid box size did not have a pronounced impact on average 
  accuracy per radii. This again is due to the size of the box being small relative to the 
  size of the spots, thus mostly not allowing the noise to play a part in the calculation of the 
  centre of the spot. One difference that the noisy data does seem to show is that it gives a more 
  even spread of answers around it's average, this suggests that although the noisy data 
  is not changing the average it is having an effect on the centroiding algorithm. The limiting factor 
  in this case is the box size in which the algorithm is interested in.

  In figure \ref{fig:box_11} the box size used was 11x11, this increases the sub-pixel localisation 
  drastically compared to figure \ref{fig:box_3}. The spread of all answers is smaller using 11x11 instead of using 3x3
  boxes, the spread also changes with radii sequentially getting larger along with the average. 
  This is due to the larger box size being able to fit 
  more pixels in to average over effectively, this is clearly evident in the R5.66 and R8.00 histogram as there is a 
  significant jump in inaccuracy as the radii is still larger than the radii of the box. 

  In figure \ref{fig:box_11_noise} when centroiding was used on the noisy data with an 11x11 box size, 
  on average the accuracy decreased as expected and gave a similar average error for each radii apart from R8.00. 
  Again this will be due to the fact that the R8.00 spot will not fit into the box size being calculated 
  but also in relation to the spot, as the size of the spot increases so to does the noise in the images 
  provided.

  In figure \ref{fig:box_var_r8} the box size for making the centroiding calculation 
  was varied from 3 to 49 $\textrm{pixels}^2$ in odd increments, this was tested to understand the 
  relationship between box size and accuracy. In general the larger the box size the more accuracy 
  gained this was expected especially for the perfect data as there was only one spot per image, 
  although this would be impractical for real data as spot are more than likely to be closer than 
  $\approx 15$ pixels. 

  Figure \ref{fig:noise_box_r8}, similar to figure \ref{fig:box_var_r8}, shows the accuracy of the 
  centroid method against the varying box size, but on a set of noisy data. Here it can be seen that 
  for a period of box size increases the accuracy also increases, then it minimises and then decreases indefinitely.
  Presumably this specific box size minimises the error at $\approx 15$ pixels due to the radii being R8.00, 
  as each spot seems to be most optimised at double it's radii. This makes sense as if the centre is 
  what the algorithm is looking for then it should look exactly where that spot is ideally, presuming 
  there are other artefacts in the image like noise. 
